\documentclass[11pt,a4paper]{article}

\usepackage{amsmath}
\usepackage{amssymb}
\usepackage{parskip}
\usepackage{fullpage}
\usepackage{hyperref}
\usepackage[linesnumbered,ruled,vlined]{algorithm2e}
\usepackage{algpseudocode}
\usepackage[margin=0.60in]{geometry}
\usepackage{fancyhdr}
\usepackage{graphicx}
\pagestyle{fancy}
\fancyhead[RO]{Ferrulli Massimiliano}
\fancyfoot{}

\usepackage{titling}
\setlength{\droptitle}{-1.5cm}

\newcommand{\Lim}[1]{\raisebox{0.5ex}{\scalebox{0.8}{$\displaystyle \lim_{#1}\;$}}}

\SetKwInput{KwData}{Input}
\SetKwInput{KwResult}{Output}
\RestyleAlgo{ruled}


\hypersetup{
    colorlinks=true,
    linkcolor=black,
    urlcolor=blue,
    pdftitle={1},
    pdfpagemode=FullScreen,
}

\title{Documentation for Project Sinussum}
\author{Ferrulli Massimiliano}
\date{}



\begin{document}

\maketitle

\section{Outcome of the analysis phase}
After creating a matrix filled with spaces, labeled as "empty\_matrix", I fill the time axis with the char "." . This is possible with the function "assign\_time\_axe" that calls "time\_index\_i", this last function determines the row index
i at which the time axis should be inserted into the matrix. Then, based on the signal input, "matrix\_chosen" determines which theoretical values needs to be computed by calling one of the three theoretical functions (\underline{SIGNAL}\_theory\footnote{The underlined word \underline{SIGNAL} is used to not repeat the three signals, SQUARE,TRIANGLE and SAWTOOTH.}).
These three functions are responsible for computing values for each cell along the temporal axis and assigning them to the input matrix using the '+' character. The same steps are done to calculated the approximated values but the assigned character is '*', the functions I use for this part are "functions\_approx", "approximated\_matrix\_chosen", "value\_\underline{SIGNAL}\_approx". 
The function that returns the row index i for every calculated value is "row\_index". In order to print the graph on the terminal, every element of the matrix are printed row by row using "print", the horizontal bars are printed with "print\_bars". 
For the dichotomic research of the maximum a function called "time\_dichotomy" returns a vector containing in the first slot the t\_start and in the second the t\_finish, based on the signal input. The dichotomic research is made by using the iterative function "max\_dicho\_research" which selects, every iterations, in which side of the graph the research must continue.





\section{Order of complexity of task 2}
All "TYPE\_theory" functions exhibit complexity $O(nbC) = O(nbL)$ ($nbC = 2 \cdot nbL - 1$). "functions\_approx" has complexity $O(nbL \cdot nbN)$ due to its for loop, which increments t from tmin to tmax with delta\_t, requiring nbC iterations ($O(nbL)$). Within this loop, the "approximated\_matrix\_chosen" function is invoked with a complexity of $O(nbN)$ based on the sum of the fourier's formulas. The "print" function has a complexity $O(nbL^2)$ due to the two for loops, the final complexity is $O(nbL \cdot nbN ) + O(nbL^2) $. The bigger terms between $nbN$ and $nbL$ will determinate which of the two sides of the sum will be the complexity.  
\section{Pseudocode for the dichotomic research of the maximum for the signal SQUARE (2nd Page) }
%
%\begin{algorithm}
%    \caption{Maximum research for SQUARE}\label{alg:two}
%    \KwData{$a,b,\varepsilon, nbN$}
%    \KwResult{$\text{Max\_Square}_\text{approx}$}
%    $c \gets 0$\\
%    $c_\text{previous} \gets 0$\\
%    %$N \gets n$\;
%    \While{$|f(c) - f(c_\text{previous})| \geq \varepsilon$}{
%      \vspace{2mm}
%      $c_\text{previous} \gets c$ \\
%      $c \gets \frac{a+b}{2}$\\
%      $f(a) \gets \text{value\_square\_approx}(a,nbN)  $ \\
%      $f(t_{finish}) \gets \text{value\_square\_approx}(b,nbN)  $ \\
%      $f(c) \gets \text{value\_square\_approx}(c,nbN)  $ \\
%      $f(c_\text{previous}) \gets \text{value\_square\_approx}(c_\text{previous},nbN) $ \\
%      \vspace{2mm}
%      \eIf{$f(c) < f(a) \, and \,  f(c) > f(t_{finish}) $}{
%        $b \gets c$ \Comment*[r]{This is a comment}
%      }
%      \eIf{$N$ is odd}{
%          $y \gets y \times X$\;
%          $N \gets N - 1$\;
%      }
%    }
%   \end{algorithm}



\begin{algorithm}
    \DontPrintSemicolon % Some LaTeX compilers require you to use \dontprintsemicolon instead
    %\KwIn{$t_{start},t_{finish},\varepsilon, nbN$  where $t_{start},t_{finish} \in \mathbb{R}^+$ \, , \, $a < b$ \, , \, $\varepsilon , nbN \in \mathbb{N^*}$ }
    \KwIn{$t_{start},t_{finish},\varepsilon \, $ three decimal numbers, where: $t_{start} < t_{finish}$  ,  $\varepsilon \in \mathbb{R}^+ $ \, and \, $t_{start},t_{finish} \geq 0$ , an integer $ nbN > 0$}
    \KwOut{The maximum value approximated for the function SQUARE \\ \hspace{18mm}in the interval $\{t_{start},t_{finish}\}$ }
    $c \gets 0$\\
    \While{$|f(c) - f(c_\text{previous})| \geq \varepsilon$}{
      \vspace{1.5mm}
      $c_\text{previous} \gets c$ \\
      $c \gets \frac{t_{start}+t_{finish}}{2}$\\
      $f(t_{start}) \gets \text{value\_square\_approx}(t_{start},nbN)  $ \\
      $f(t_{finish}) \gets \text{value\_square\_approx}(t_{finish},nbN)  $ \\
      $f(c) \gets \text{value\_square\_approx}(c,nbN)  $ \\
      $f(c_\text{previous}) \gets \text{value\_square\_approx}(c_\text{previous},nbN) $ \\
      \vspace{2mm}
      % The "u" before the "If" makes it so there is no "end" after the statement, so the else will then follow
      \uIf{$f(c) < f(t_{start}) \, and \,  f(c) > f(t_{finish}) $}{
        $t_{finish} \gets c$
      }
      \uIf{$f(c) > f(t_{start}) \, and \,  f(c) < f(t_{finish}) $}{
        $t_{start} \gets c$
      }
      \uIf{$ (f(c) < f(t_{start}) \, and \,  f(c) < f(t_{finish})) \, \,  or \, \, (f(c) > f(t_{start}) \, and \,  f(c) > f(t_{finish}))  $}{
         \uIf{$f(t_{finish}) < f(t_{start}) $}{
            $t_{finish} \gets c$
         } 
         \uIf{$f(t_{finish}) > f(t_{start}) $}{
            $t_{start} \gets c$
         }        
      }
      \uIf{$(f(c) = f(t_{start}) )  \, \,  or \, \, (f(c) = f(t_{finish})) $}{
        \uIf{$f(t_{finish}) < f(t_{start}) $}{
           $t_{finish} \gets c$
        } 
        \uIf{$f(t_{finish}) > f(t_{start}) $}{
           $t_{start} \gets c$
        }        
     }
    }
    \Return{f(c)}
    \caption{{Max dicho research}}
    \label{algo:duplicate}
    \end{algorithm}

\section{Behaviour for $ \Lim{nbN \to \infty}$   }
While increasing nbN I notice that for the approximated maximum for the signal TRIANGLE tends to 1 while for SQUARE and SAWTOOTH the maximum tends\footnotetext{For SQUARE 1.17897974, for SAWTOOTH 1.17897874} to 1.17897 (value obtained with $nbN = 10^6$ and $nbN = 10^7$) 
\end{document}